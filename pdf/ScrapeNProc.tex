\documentclass[]{beamer}
% Class options include: notes, notesonly, handout, trans,
%                        hidesubsections, shadesubsections,
%                        inrow, blue, red, grey, brown
%\usetheme[height=9mm]{Rochester}
\usetheme{Frankfurt}
\usepackage{booktabs}
\renewcommand{\arraystretch}{1.2}

\usepackage{url}
\renewcommand\UrlFont{\ttfamily\color{blue}}%
\usepackage{hyperref}
% \renewcommand\url{%

%  \renewcommand\UrlLeft{\uline\bgroup}%
%  \renewcommand\UrlRight{\egroup}}


% Other themes include: beamerthemebars, beamerthemelined,
%                       beamerthemetree, beamerthemetreebars

\title{Part I: Acquiring Data from the Web}    % Enter your title between curly braces

\author{Solomon Messing}                 % Enter your name between curly braces
\author[]{Solomon Messing}
\institute{Department of Communication, Statistics\\ Stanford Social Science Data and Software (SSDS)}%\\ Facebook Data Science}
\date{\today}                    % Enter the date or \today between curly braces

\begin{document}

% Creates title page of slide show using above information
\begin{frame}
\titlepage
\end{frame}

%\section[Outline]{Outline}
%\begin{frame}
%  \tableofcontents
%\end{frame}

\section{Motivation}
\setcounter{subsection}{1}

\begin{frame}
\frametitle{Overview}   % Insert frame title between curly braces
\begin{itemize}
  \item How to get data from the web (cURL, APIs, JSON, XML)
  \item Extracting useful stuff from the results (Regex)
  \item Representing text as data
  \item Getting meaningful features (Regex, R)
  \item Document-term matrices
  \item Supervised and unsupervised approaches to analysis
\end{itemize}
\end{frame}

\begin{frame}
\frametitle{What to expect}   % Insert frame title between curly braces
\begin{itemize}
  \item Goal: Quantitative insight from haystack of messy data
  \item Labs: programming, lots of R!
  \item Working with your neighbors
  \item Adapting starter code + code found on Google, StackOverflow, etc.
  \item You learn better, I talk less, you have code to work with.
\end{itemize}
\begin{center}
\begin{tabular}{cc}
\includegraphics[height=1in]{images/glider.png} &
\includegraphics[height=1in]{images/hacksaw.jpeg}
\end{tabular}
\end{center}
\end{frame}

\begin{frame}
\frametitle{Motivation}   % Insert frame title between curly braces
\begin{itemize}
  \item Massive growth in availability of text and other data
  \item 100K Tweets per min, 700K FB shares per min (DOMO), 200M emails---10 min = 1 LOC (Huggins)
  \item Proliferation of structured/semi-structured data at your fingertips: open-data, APIs, and scrape-able data sources
  \item Growth of open-source tools to acquire and analyze this data
\end{itemize}
\begin{center}
\includegraphics[height=1in]{images/FB_image.png}
\end{center}
\end{frame}

\begin{frame}
\frametitle{Motivation: what's out there?}   % Insert frame title between curly braces
\begin{itemize}
  \item Raw text, html tables, semi-structured HTML/XML.
  \item Data marketplaces, e.g.: \small\url{http://www.infochimps.com/datasets}
  \item APIs:
  \begin{itemize}
    \begin{small}
\item \url{https://dev.twitter.com/}
\item \url{http://developer.nytimes.com/} articles, campaign finance, congress, entities, geography/population
\item \url{http://developer.washingtonpost.com/} political speeches, campaign finance, White House visitors log
\item \url{http://developer.yahoo.com/everything.html} - search, finance, geo-coding, on-the-fly entity extraction, content analysis, term extraction.
\item \url{https://api.facebook.com} - access (a little) Facebook data.\footnote{e.g., \href{https://api.facebook.com/method/fql.query?query=select\%20total_count,like_count,comment_count,share_count,click_count\%20from\%20link_stat\%20where\%20url='http://www.nytimes.com/2012/09/22/opinion/voter-harassment-circa-2012.html'&format=json}{\color{blue} get aggregate likes for NYTimes.com article}}
\end{small}
  \end{itemize}
\end{itemize}
\end{frame}

\section{Get data}
\setcounter{subsection}{1}
\begin{frame}
\frametitle{Data from the Web}   % Insert frame title between curly braces
\begin{enumerate}
  \item Hit a server
  \item Parse it's response
  \item Clean and transform into something useful
  \item Often by merging it with something else.
\end{enumerate}
\end{frame}

\begin{frame}
\frametitle{Hit a server}   % Insert frame title between curly braces
\includegraphics[height=1.5in]{images/complex_url.png}
~\\ \small{from: http://doepud.co.uk/blog/anatomy-of-a-url.php}
\end{frame}

\begin{frame}
\frametitle{Hit a server, in *.NIX}   % Insert frame title between curly braces
\includegraphics[height=1.5in]{images/unix.jpg}
\includegraphics[height=1.5in]{images/unix2.jpg}
~\\ \small{Unix/Linux was made for this.  Windows was not.}
\end{frame}

\begin{frame}
\frametitle{Hit a server with cURL}   % Insert frame title between curly braces
\begin{itemize}
  \item Use cURL (RCurl) to send GET or POST request to server for a URL/URI
  \item \verb~curl http://thecaucus.blogs.nytimes.com~
  \item \verb~curl -L http://t.co/KtxsapBV~
  \item \verb~curl http://search.twitter.com/search.json?q=\%40obama~
  \item The latter is a query string, can used to return custom results from MANY websites (Twitter, nytimes.com, Library of Congress, etc.).
  ~\\~\\ \tiny{Query string trivia: the following string has brought down many a web server:\\
  \texttt{system\%28\%27rm+-rf+\%2F\%27\%29}}
\end{itemize}
\end{frame}

\begin{frame}
\frametitle{Hit a server (or an entire site!)}   % Insert frame title between curly braces
\begin{itemize}
  \item Use wget or a crawler
  \item \verb~wget http://thecaucus.blogs.nytimes.com~
  \item See also module 3 \url{http://www.stanford.edu/~seanjw/module3/\#8}
  \item Heritrix \url{http://crawler.archive.org/}
\end{itemize}
\end{frame}

\begin{frame}
\frametitle{Hit a server (on schedule!)}   % Insert frame title between curly braces
\begin{itemize}
  \item Use \verb~cron~
  \item \verb~crontab -e~
  \item in VIM type \verb~ * * * * * /4 /path/to/R CMD myfile.R~ to run every Wed
  \item or perhaps \verb~ * * * * * /4 /path/wget http://thecaucus.blogs.nytimes.com~
  \item Save and you'll see: \verb~crontab: installing new crontab~
  \item Type \verb~crontab -l~ to see your crontabls
  \item See \url{http://benr75.com/pages/using_crontab_mac_os_x_unix_linux} for more.
\end{itemize}
\end{frame}

\begin{frame}
\frametitle{Parse the server's reply: JSON}   % Insert frame title between curly braces
\begin{itemize}
  \item What the **** is JSON?
  \item JSON: Java script object notation
  \item for serializing objects, for the purposes of data interchange
  \item semi-structured, tree-like, and pretty simple
  \item Nice JSON viewer for \href{https://chrome.google.com/webstore/detail/chklaanhfefbnpoihckbnefhakgolnmc}{\color{blue} Chrome} and \href{https://addons.mozilla.org/en-US/firefox/addon/jsonview/}{\color{blue} Firefox}---Use this in the R code.
  \item Sean's overview from module 3 \url{http://www.stanford.edu/~seanjw/module3/\#42}
  \item Widom's overview of JSON here: \url{http://www.db-class.org/course/video/preview_list}
\end{itemize}

\end{frame}

\begin{frame}
\frametitle{Record and parse it: JSON}   % Insert frame title between curly braces
\begin{columns}[onlytextwidth]
\begin{column}{.5\textwidth}
\centering
\includegraphics[height=2.3in]{images/JSON.png}
\end{column}
\begin{column}{.4\textwidth}
\begin{itemize}
  \item Base values\\
  \item Objects \{\}  - label-value pairs\\
  \item Arrays [] - list of values\\
  \item nested sets of arrays - not a table\\
  \item NO SCHEMA\\
  \item NO SQL (hard to query)\\
\end{itemize}
\end{column}
\end{columns}
\end{frame}

\begin{frame}
\frametitle{Record and parse it: XML}   % Insert frame title between curly braces
\begin{itemize}
  \item What the **** is XML?
  \item XML: eXtensible Markup Language
  \item Like HTML, but tags describe data, not formatting
  \item Semi-structured, tree-like, much richer, more complicated than JSON
  \item Nice XML viewer for \href{https://chrome.google.com/webstore/detail/eeocglpgjdpaefaedpblffpeebgmgddk}{\color{blue} Chrome}.
  \item Sean's overview in module 3 \url{http://www.stanford.edu/~seanjw/module3/\#21}
  \item Widom's course: \url{http://www.db-class.org/course/video/preview_list}
\end{itemize}
\end{frame}

\begin{frame}
\frametitle{Record and parse it: XML}   % Insert frame title between curly braces
\begin{columns}[onlytextwidth]
\begin{column}{.5\textwidth}
\centering
\includegraphics[height=1.2in]{images/XMLtree.jpg}\\ \tiny Looks like:\\~\\
\includegraphics[height=1.2in]{images/XML.png}
\end{column}
\begin{column}{.4\textwidth}
\begin{itemize}
  \item Tagged elements
  \item Attributes
  \item Text
  \item Nested structure - not a table\\
  \item XML SCHEMA/DTD\\
  \item NO SQL (use XPATH/jQuery)\\
\end{itemize}
\end{column}
\end{columns}
~\\\tiny Example from \url{http://www.w3schools.com/xml/xml_tree.asp}.
\end{frame}

%\begin{frame}
%\frametitle{Lab 1}   % Insert frame title between curly braces
%\center \url{https://dl.dropbox.com/u/25710348/CSSscraping/scripts/CurlJSONXML.R}
%\end{frame}

\begin{frame}
  \frametitle{Lab 1}   % Insert frame title between curly braces
	\href{https://dl.dropbox.com/u/25710348/CSSscraping/scripts/CurlJSONXML.R}{\color{blue} Lab 1: Getting useful data with Curl and JSON in R}.
\end{frame}

%%%%%%%%%%%%%%%%%%%%%%%%%%%%%%%%%
%%%%%%%%%%%%%%% REGEX %%%%%%%%%%%
%%%%%%%%%%%%%%%%%%%%%%%%%%%%%%%%%

\section{Extract/clean}
\setcounter{subsection}{1}
\begin{frame}
\frametitle{Clean things up: Regular Expressions}   % Insert frame title between curly braces
\center
\includegraphics[height=2.5in]{images/regular_expressions.png}
~\\ \small{from: https://xkcd.com/208/}
\end{frame}

\begin{frame}
\frametitle{Regex in action..}   % Insert frame title between curly braces
\begin{itemize}
  \item To clean up data after scraping \url{http://www.r-bloggers.com/scrape-web-data-using-r/}.
  \item To extract useful information (state, latitude, longitude), when scraping a web page \url{http://solomonmessing.wordpress.com/2011/09/18/map-of-participants/}
  \item \texttt{grep} to create custom indicator variable features for analysis, \texttt{agrep} for approximate version of 		\
texttt{grep}. More on this later.
\item For nice reference materials, look to \url{http://www.regular-expressions.info/reference.html}.
\end{itemize}
\end{frame}

\begin{frame}
\frametitle{RegEx bare essentials}   % Insert frame title between curly braces
\begin{tabular}{lp{1.5in}p{2in}}
RegEx & Description & Example \\ \hline
bil & Match string `bil' & this is a 2 dollar \underline{bil}l. \\
dollar$|$bill & Match dollar or bill & this is a 2 \underline{dollar} \underline{bill}. \\
\textbackslash d & Match any digit & this is a \underline{2} dollar bill. \\
\textbackslash w & Match any word (single letter) & \underline{this} \underline{is} \underline{a} 2 \underline{dollar} \underline{bill}. \\
\textbackslash w+ & Match at least 2 letters & \underline{this} \underline{is} a 2 \underline{dollar} \underline{bill}. \\
\textbackslash s & Match any whitespace & this\underline{ }is\underline{ }a\underline{ }2\underline{ }dollar\underline{ }bill. \\
\textbackslash S & Match NOT whitespace & \underline{this} \underline{is} \underline{a} \underline{2} \underline{dollar} \underline{bill}.\\
colo?ur & Optionally match character preceding `?' & Yanks \underline{color}, Brits \underline{colour}. \\
col.*r & match any character between l and r 0 or more times & Yanks \underline{color}, Brits \underline{colour}. \\
\end{tabular}
\end{frame}

\begin{frame}
\frametitle{Regex in R for Cleaning and Feature Extraction}   % Insert frame title between curly braces
\footnotesize
\begin{tabular}{p{2.5in}p{1.5in}}
Command & What it does \\ \hline
\texttt{grep("dollar$\backslash\backslash|$bill", moneyStuff)} & return index of \texttt{moneyStuff} with item.\\
\texttt{gregexpr("$\backslash\backslash$d", "4a53f45e")} & return index of string in each match (why might this be a bad idea?)\\
\texttt{str\_extract(moneyStuff, "$\backslash\backslash$d+")} & extracts groups of digits \\
\texttt{str\_replace(moneyStuff, "($\backslash\backslash$d+)", "$\backslash$\$ $\backslash\backslash$1)")} & inserts dollar sign in front of numbers \\
\texttt{agrep("dollar$\backslash\backslash|$bill", moneyStuff, max.distance = .2)} & return index of anything with edit distance $\leq$ .2 from dollar or bill in \texttt{moneyStuff} with item.  VERY useful to handle misspellings, plagiarism, etc.\\~\\
\small Read up on edit distance here: \url{http://www.stanford.edu/class/cs124/lec/med.pdf}
\end{tabular}
\end{frame}

\begin{frame}
\frametitle{Regex == AWK}   % Insert frame title between curly braces
Need to quickly extract/transform a 2 TB text file? \\
\begin{itemize}
  \item \verb~awk '/pattern/'~ return every line in a file matching pattern\\
  \item \verb~cat bigfile.csv | awk '($2>5 \&\& $3<2) \{print $1,$3\}' > smallerfile.csv~\\
  Reads in all lines from bigfile.csv where 2nd column value $>5$  and third column $<2$ and prints to smallerfile.csv
\end{itemize}
\begin{center} \includegraphics[height=.8in]{images/awk.png} \end{center}
\end{frame}

\begin{frame}
\frametitle{Lab 2}   % Insert frame title between curly braces
\center \href{https://dl.dropbox.com/u/25710348/CSSscraping/scripts/RegexCurl.R}{\color{blue} Lab 2: Scraping and Regex}
\end{frame}


\end{document}
