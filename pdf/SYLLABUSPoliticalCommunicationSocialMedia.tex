\documentclass[12pt]{article}
\usepackage[linkcolor=blue,
colorlinks=true,
urlcolor=blue,
citecolor=blue,
pdftitle={Political Communication and Social Media }]{hyperref}
\usepackage{url}

\usepackage{booktabs}
\usepackage{setspace}
\singlespacing
\usepackage{graphicx}

% Needed for in line references:
%\usepackage{natbib}
\usepackage[square,sort,comma,numbers]{natbib}
\usepackage{bibentry}
%\newcommand{\bibverse}[1]{\begin{verse} \bibentry{#1}. \end{verse}}
\newcommand{\bibverse}[2][]{\begin{verse} \bibentry{#2}. #1\end{verse}}

\textwidth=7in
\textheight=9.5in
\topmargin=-1in
\headheight=0in
\headsep=.5in
\hoffset  -.85in

\pagestyle{empty}

\renewcommand{\thefootnote}{\fnsymbol{footnote}}
\begin{document}

% Needed for in line references:
\nobibliography*

\begin{center}
{\bf COMM 102S: Political Communication and Social Media\\
TTH 2:15-3:30 PM\\
Chaffee Seminar Room (Building 120, Room 452) 
}
\end{center}

\setlength{\unitlength}{1in}

\begin{picture}(6,.1) 
\put(0,0) {\line(1,0){6.25}}         
\end{picture}


\noindent THIS SYLLABUS WILL CHANGE.  CHECK \href{http://dl.dropbox.com/u/25710348/PoliticalCommunicationSocialMedia/SYLLABUSPoliticalCommunicationSocialMedia.pdf}{HERE} FOR UPDATES. 

% Basic info about the course
%\vskip.25in
\vskip.15in
\noindent \textbf{Instructor}: Solomon Messing\\
\textbf{Email}: \url{[lastname]@stanford.edu}\\%Phone: (858) 357-6735
\textbf{Web}: \url{www.stanford.edu/~messing}\\
%\vskip.25in
\noindent\textbf{Office}: Building 120 (McClatchy Hall) Room 444\\
\textbf{Office Hours:} TH 12:00-2:00, and by appointment.



\setkeys{Gin}{width=0.95\textwidth}
\begin{figure}[h!tb]
\centering
\vspace{.1cm}
\includegraphics{SyllabusPics.jpg}
\end{figure}  



%\renewcommand{\arraystretch}{2}

% Course summary
%\vskip.15in
\noindent \textbf{Summary}: This course examines the role of social technologies in modern politics and electoral campaigns. We will examine the social science literature surrounding media and politics, then formulate arguments about how the social web, mobile technologies, and large scale (``big data'') analytics change the nature of political communication and representation.  We will strive to answer the following questions: which theorized cause-and-effect relationships do these technologies ``disrupt?''  Which relationships still hold?  Why?  First, we will explore the structure of costs and benefits for ordinary citizens, candidates, and activists in politics and campaigns, and think about how the use of these technologies changes the incentives and strategic possibilities for each of these actors.  We will use the 2008 and 2012 elections as case studies of digital campaign strategy and voter behavior.  We will then examine the role of social media in connecting constituents to their representatives, ask how voters respond to such campaigns, and close by discussing what social media means for measuring public opinion.  The course will culminate with a final project designed to strengthen your data analysis skills and writing portfolio.
\newpage

\noindent \textbf{Course Objectives}: By the end of this course you will be able to (1) inform substantive theoretical questions with data analysis (labs and problem sets); (2) be able to collect and analyze social media data to answer questions about politics (final project); and (3) comment on \emph{how} social technologies should be expected to change the conduct of both American and international politics, if at all.  
\vskip.15in

\noindent \textbf{Course Outline}: 

\begin{center} \begin{minipage}{6in}
\begin{flushleft}

Introduction \& Social Science Review \dotfill Week 1\\ 

\vspace*{.05in} \textbf{Campaigns: Data, Social Media, \& Mobile}\\ \vspace*{.05in}
The Media Campaign \dotfill Week 2\\
Questioning the Media Campaign \dotfill Week 2\\
\vspace*{.05in} \textbf{\href{https://dl.dropboxusercontent.com/u/25710348/COMM102S/psets/PSET1.pdf}{Problem Set 1} Due \dotfill Week 2} \\ \vspace*{.05in}
Back to the Future? Campaign Machinery in the Digital Age \dotfill Week 3\\
Campaigns of the Future: Data, Social Science, \& Social Media \dotfill Week 3\\
\vspace*{.05in} \textbf{Representatives \& Constituent Communication} \\
Theories of Representation \& Constituent Communication \dotfill Week 4\\ 
Representation \& Constituent Communication on Twitter \dotfill Week 4\\ 
\vspace*{.05in} \textbf{\href{https://dl.dropboxusercontent.com/u/25710348/COMM102S/psets/PSET2.pdf}{Problem Set 2} Due \dotfill Week 4} \\ \vspace*{.05in}

\vspace*{.05in} \textbf{The Public, the Media, \& the Social/Mobile Web} \\ \vspace*{.05in}
The Media Agenda \dotfill Week 5\\
What is ``New Media''? \dotfill Week 5\\
\vspace*{.05in} \textbf{\href{http://dl.dropboxusercontent.com/u/25710348/PoliticalCommunicationSocialMedia/ProjectProposalPCSM.pdf}{Project Proposals} Due \dotfill Week 5} \\ \vspace*{.05in}
The Reasoning Voter? \dotfill Week 6\\
%The Information Environment \dotfill Week 6\\
Selective Exposure to Media and Campaigns \dotfill Week 6\\
Polarization \dotfill Week 7\\
\vspace*{.05in} \textbf{Social Media and the Pulse of the Nation}\\ \vspace*{.05in}
%Digital Protest \dotfill Week 8\\
Public opinion and the democratic process \dotfill Week 8\\
Does social media reflect public opinion? Define it?  \dotfill Week 8\\
%Hacktivism \dotfill Week 8\\
\vspace*{.05in} \textbf{Final Project Due \dotfill Week 8} \\ \vspace*{.05in}

\end{flushleft}
\end{minipage}
\end{center}


% Prerequisites
\vskip.25in
\noindent\textbf{Prerequisites:} I will assume introductory social science knowledge, including a familiarity with problems related to measuring social phenomena, correlation versus causation, and an understanding of statistics (e.g., STATS 60; plus one or more of PSYCH 252, POLISCI 350A\&B, EDUC 260X/STATS 209).  I will also assume programming experience equivalent to CS 106 and CS 109.  If you are not sure that you meet these prerequisites do not hesitate to contact me.  %\footnotemark 
%\footnotetext{
%Give the following texts a look if you are unsure that you are adequately prepared:\\

\vskip.25in
\noindent\textbf{Course Requirements and Grading}: Your grade will be based on responses to the readings (10\%) due each week, participation in class and in the online discussion boards (10\%), problem sets (20\%), your project proposal due at the end of week 5 (20\%), and your final project (40\%).  The final project should address the role of a particular social technology in campaigns, representation, political behavior, or activism.  It should (1) summarize what is known about the issue, (2) identify an open question, and (3) then present relevant evidence (quantitative data) that sheds new light on that issue.  The deadline to submit a paper proposal is the end of week 5, but I strongly suggest you start \emph{much} earlier. 

\vskip.25in
\begin{center}
\begin{tabular}{rcc}
\toprule
What & Weight & When \\ 
\midrule
Class Participation/Online Discussion & 10\% & -- \\ 
Reading Responses & 10\% & -- \\ 
Problem Sets & 20\% & End of Weeks 2 and 4 \\ 
Project Proposal & 20\% & End of Week 5 \\ 
Final Project & 40\% & End of Week 8 \\ 
\bottomrule
\end{tabular}
\end{center}
\vskip.25in

\noindent\textbf{Class Participation}: Our class meetings should be highly interactive, so attendance and participation are mandatory for every class meeting.  Please come prepared to discuss the readings and participate in the labs.  
\vskip.15in

\noindent\textbf{Labs}: We will have periodic data analysis labs.  I strongly recommend you use R for this lab so that you can take advantage of the starter code I provide and R's capabilities.  If you prefer, you may use Python.  An introduction to R for the beginner is available \href{https://dl.dropboxusercontent.com/u/25710348/Blogposts/RIntro.R}{here}. 
\vskip.15in

\noindent\textbf{Weekly Responses}: Readings consist of a mix of formal social science academic journal articles, excerpts from book chapters, media coverage and video clips.  After engaging with these materials, submit a brief writeup linking the work in the labs or problem sets to the readings.  These will be due by 9am each Thursday of class.  Responses will be graded as +2 (Outstanding, offer particularly novel insights), +1 (Satisfactory, indicates you've done the readings and labs, and have applied some thought to them) or +0 (Unsatisfactory).

\vskip.05in
\noindent Please compose your weekly responses in a Google Doc, and submit them via \href{https://docs.google.com/forms/d/1TKhmE5hY2ZKjYfgy-mq5CUxnAjfs61Pa9IbweVuHeDU/viewform}{this form}.  Remember to give them a thorough edit before submitting so you get maximum points. 
\vskip.15in

\noindent\textbf{How to Prepare for Reading Responses/Participation in Class}: First read the abstract, then read the discussion/conclusion you have an idea of the argument or the statements and concepts the author is using.  Then attempt to determine what evidence the author is using to back these claims---the meat of the paper.  Are the authors making an argument that merely describes the world or are they making a causal claim?  If it's a causal claim, what exactly is the claim?  And what is the research design---observational, experimental, qualitative/case study research?  Think about whether the concepts the authors outline are consistent with the empirical evidence the authors present.  What might confound the results?  How could the study be improved?  Reading by attempting to answer these questions, while carefully evaluating the authors' evidence should prepare you to achieve an A in this course.  It will help prepare you for success outside of this course as well.  
\vskip.25in

\noindent\textbf{Final Project}:  There is no final exam for this class.  Rather, the final project will allow you to explore open questions in the field and get experience working with data.  You'll need to identify a specific research question addressing some aspect of how candidates use social media to campaign, or how representatives and constituents communicate using social media.  You can (1) use data to describe how these processes work, or (2) formulate expectations/hypotheses about patterns in the data we would expect if your hypothesis is true, and test it using actual data.  You will be graded based on the quality of your work, not whether you successfully confirm your hypotheses.  I recommend you use the Twitter data set that I've put together for this class, though you may use other data sources as well---please speak with me if you'd like to use additional data or do something completely different.  \textbf{The goal for the final project is to strengthen your data analysis skills and your writing portfolio (and so hopefully your applications to top-tier graduate or undergraduate university research programs).}  
\vskip.10in

\noindent\textbf{Project Proposal}: You'll need to propose a research project by \textbf{Week 5}.  This proposal will consist of all sections that will comprise the final write-up (Introduction, Proposed Methods, Expected Results, Discussion).  Proposals are expected to be 4-5 double-spaced pages.  This is meant to form the basis of your final project and to make sure you are not distributing the work unevenly.  Ideally, you should be able to plug in a few figures, turn your ``Expected Results'' section into a ``Results'' section, modify your discussion a bit, then hand in your final paper.   \href{http://dl.dropboxusercontent.com/u/25710348/PoliticalCommunicationSocialMedia/ProjectProposalPCSM.pdf}{How-to hand-out for writing the proposal available here}.

\vskip.10in
\noindent\textbf{Final Project Deliverables}: The final project will extend your proposal and present actual results.  This proposal will consist of the following sections: Introduction, Methods, Results, Discussion.  It should be anywhere from 10-20 pages (double-spaced), the majority of which should consist of results.  
\vskip.10in

\noindent\textbf{How to Do Well on the Final Project}: Get feedback from me and other students early and often.  Make sure to read over and edit both your proposal and your final project a number of times before you submit.  Papers edited multiple times generally address questions that are framed to be more relevant to your readers, have more refined arguments, clearer evidence, and ultimately higher grades.  This will also improve your writing skills, which will be useful far beyond this course. 
\vskip.10in

\noindent\textbf{Examples of papers from past students}:
%\vskip.05in
~\\\href{https://www.stanford.edu/class/polisci120b/readings/samples/socialmedia.pdf}{Social Media: Emerging Leaders or Followers?}\\
%\href{https://www.stanford.edu/class/polisci120b/readings/samples/issue.pdf}{Issue Ownership and Internet Search Traffic}\\
%\href{https://www.stanford.edu/class/polisci120b/readings/samples/predicting.pdf}{Predicting Presidential Elections}\\
\href{https://www.stanford.edu/class/polisci120b/readings/samples/editorial.pdf}{The Newspaper Editorial Endorsement}\\
%\href{https://www.stanford.edu/class/polisci120b/readings/samples/turnout.pdf}{The Effects of Negative Advertising on Voter Turnout}\\
\href{https://www.stanford.edu/class/polisci120b/readings/samples/palin.pdf}{Sarah Palin and Gender}\\

\vskip.25in
\noindent\textbf{Websites:}  The most current version of the syllabus (this document) will be maintained \href{https://dl.dropbox.com/u/25710348/PoliticalCommunicationSocialMedia/SYLLABUSPoliticalCommunicationSocialMedia.pdf}{here}. All required readings not available on the internet will be made available on \href{http://coursework.stanford.edu}{coursework}.  We will use the \href{https://piazza.com/}{piazza} forum as a discussion platform.  Finally, we will use \href{https://www.zotero.org/}{zotero} for bibliography/citation management.  On the first day of class, please sign up for piazza and zotero accounts, and request to join the zotero group ``StanfordComm.''  There is a hot folder in the group library called ``SocialMediaPolitics,'' which we will use to build up a citation database for papers that are relevant to your final projects.  
%Access to the readings will be restricted to those enrolled or auditing the course.

\vskip.25in
\noindent\textbf{Extra Help}: Do not hesitate to come to my office during office hours or by appointment to discuss the readings, your final paper, or any aspect of the course.  For extra help finding or analyzing data, check out the \href{https://www.stanford.edu/group/ssds/cgi-bin/drupal/}{Social Science Data and Software (SSDS)}, which provides services and support to Stanford faculty, staff and students in the acquisition of social science data and the selection and use of quantitative (statistical) and qualitative analysis software.  For advice regarding data analysis, check out \href{http://www-stat.stanford.edu/consulting/index.html}{Department of Statistics drop-in consulting service}.  Also, be aware of \href{http://stackoverflow.com/}{StackOverflow}, a well-implemented and widely used programming question-and-answer site, and \href{http://stats.stackexchange.com/}{CrossValidated}, an analogous site for questions related to statistics.  I have been using and teaching R for a long time and am happy to help you figure out what you need to know for the final project if you decide to use R.  

%\vspace*{.15in}
%\noindent\textbf{Grade Policy:} In this space you can add your grading policy.


%\vskip.25in
%\noindent \textbf{Course Objectives}: A Description of the Course Objectives.
%
%This course will explore how social media and mobile computing platforms affect the modern political landscape.  Topics: how these technologies change the mix of news, information and campaign materials we get; structure our relationships with candidates and representatives; augment modern politicians' fundraising and campaign efforts; and make possible new forms of political organization and collective action.  Possible case studies: the Obama campaign's successful use of social/mobile technology to campaign in 2008 and 2012; how constituents use social media to communicate with their representatives; and the role of social-mobile technologies in modern revolutionary movements.   

\vskip.25in
\noindent\textbf{Students with Disabilities}: Students who may need an academic accommodation based on the impact of a disability must initiate the request with the Office of Accessible Education (OAE). Professional staff will evaluate the request with required documentation, recommend reasonable accommodations, and prepare an Accommodation Letter for faculty dated in the current quarter in which the request is being made. Students should contact the OAE as soon as possible since timely notice is needed to coordinate accommodations. The OAE is located at 563 Salvatierra Walk (phone: 650-723-1066, URL: \url{http://studentaffairs.stanford.edu/oae}). 

\vskip.25in
\noindent\textbf{Honor Code}: I take the Honor Code seriously. I assign writing projects because I think that they have the most pedagogical value.  Because of the nature of these projects, it will be near impossible for anyone to cheat on them, but not impossible to plagiarize.  I do not tolerate plagiarism in any form---if you are using someone else's idea, cite it; if you are borrowing someone's specific words quote and cite them.  I will pursue honor code violations to the maximum penalty allowed by Stanford rules.  

\vskip.25in
\section*{Readings and Labs}

\subsection*{Introduction \& Social Science Review}
\emph{Required:}
\bibverse[CHAPTER 1 ONLY AND MSR TALK: \url{http://bit.ly/10SQAXd}.]{kahneman_thinking_2011}

\noindent \href{http://dl.dropboxusercontent.com/u/25710348/COMM102S/ADexp.R}{\textbf{R lab}: Replicating Kahneman and Tversky's Asian Disease Experiment}.\\ 

\noindent \emph{Further reading:}
\bibverse[CHAPTER 1 ONLY (Causal Inference).]{shadish_experiments_2002}
\bibverse[URL TBD. CHAPTERS 1-2.]{ross_person_1991}
%\bibverse{freedman_statistical_1991}
%\bibverse{king_how_1986}
%\bibverse{holland_statistics_1986}
%\bibverse{pearl_introduction_2010}
\subsection*{Campaigns: Data, Social Media, \& Mobile}
\subsubsection*{The Media Campaign}
\emph{Required:}
\bibverse{polsby_consequences_1983}
\bibverse{iyengar_`basic_2000}

\noindent \emph{Further reading:}
\bibverse{simon_duck_2002}

\subsubsection*{Questioning the Media Campaign}
\emph{Required:}
\bibverse{karol_polls_2003}
\bibverse{kreiss_acting_2012}

\noindent \emph{Further reading:}
\bibverse{gerber_how_2011}

\subsubsection*{Back to the Future? The Importance of Campaign Machinery in the Digital Age}
\emph{Required:}
\bibverse{hindman_real_2005}
\bibverse{kreiss_new_2010}

\noindent \emph{Further reading:}
\bibverse[Ch. 6-13]{harfoush_yes_2009}

\subsubsection*{The Future of Campaigns: Data, Social Science, \& Social Media}
\emph{Required:}
\bibverse{cillizza_romneys_2007}
\bibverse{issenberg_how_2010}
\bibverse{issenberg_obama_2012}

\noindent \emph{Further reading:}
\bibverse{issenberg_death_2012}
\bibverse{kittur_crowdsourcing_2010}
%Fanning the Flames, Geer 

%\noindent \noindent \emph{Further reading:}

\subsection*{Representation \& Constituent Communication}
\subsubsection*{Theories of Representation \& Constituent Communication}
\emph{Required:}
\bibverse{butler_can_2011}
\bibverse{grimmer_how_2012}

\noindent \href{http://dl.dropboxusercontent.com/u/25710348/COMM102S/TwitterCongressLab.R}{\textbf{R lab}: What Constituents say to their Congressional Representatives on Twitter and Basic Text Analysis}.\\ 

\noindent \emph{Further reading:}
\bibverse{mansbridge_rethinking_2003}
\bibverse{butler_politicians_2011}
%\bibverse{butler_field_2012}

\subsubsection*{Representation \& Constituent Communication on Twitter}
\emph{Required:}
\bibverse{hemphill_whats_2013}
\bibverse{livne_networks_2011}


\noindent \emph{Further reading:}
\bibverse{tumasjan_predicting_2010}
\bibverse{roback_``id_2013}
%\bibverse{nie_world_2010}



\subsection*{The Public, the Media, \& the Social/Mobile Web}

\subsubsection*{The Reasoning Voter?}
\emph{Required:}
\bibverse[CHAPTER 3 ONLY.]{popkin_reasoning_1994}

\noindent \href{http://dl.dropboxusercontent.com/u/25710348/COMM102S/PIDStability.R}{\textbf{R lab}: The Relationship between Party ID and Vote Choice in ANES data}.\\ 

\noindent \emph{Further reading:}
\bibverse[CHAPTERS 1 \& 3.]{downs_economic_1957} 

\subsubsection*{The Information Environment}
\emph{Required:}
\bibverse{katz_two-step_1957}
\bibverse{bakshy_role_2012}

\noindent \emph{Further reading:}
%\bibverse{lewin_frontiers_1947}
%\bibverse{wu_who_2011}
%\bibverse{cha_measuring_2010}
%\bibverse{bakshy_everyones_2011}
%\bibverse{cha_measuring_2010}

\bibverse{hindman_googlearchy:_2003}

\subsubsection*{The Media Agenda and Owned Issues}
\emph{Required:}
\bibverse{behr_television_1985}
\bibverse{petrocik_issue_1996}

\noindent \emph{Further reading:}
\bibverse{gilliam_crime_1996}
%\bibverse{waldman_rhetorical_2003}

\subsubsection*{What is ``New Media''?}
\emph{Required:}
\bibverse{bennett_new_2008}
\bibverse{shirky_political_2011}

\noindent \href{http://dl.dropboxusercontent.com/u/25710348/COMM102S/MostSharedNYTimesContent.R}{\textbf{R lab}: The Difference between the Traditional Media Agenda and What People Share, a Re-Analysis of Berger and Milkman's (2012) New York Times Data}.\\ 

\noindent \emph{Further reading:}
\bibverse{messing_friends_2012}
\bibverse{bennett_personalization_2012} %personalized politics
\bibverse{mutz_communication_2011}
%\bibverse{jenkins_photoshop_2006}
%\bibverse{kwak_what_2010}
%\bibverse{bailenson_facial_2008}
%\bibverse{todorov_inferences_2005}
%\bibverse{adamic_political_2005}
%\bibverse{oreilly_what_2007}
%\bibverse{stepanek_cause_2010} %filtering

\subsubsection*{Selective Exposure to Media and Campaigns}
\emph{Required:}
\bibverse{iyengar_selective_2008}
\bibverse{messing_selective_2012}

\noindent \href{http://dl.dropboxusercontent.com/u/25710348/COMM102S/MessingWestwood2011.R}{\textbf{R lab}: Re-analyzing Messing and Westwood (2012) and the importance of the independence assumption}.\\ 

\noindent \emph{Further reading:}
\bibverse{iyengar_red_2009}

\subsubsection*{Polarization}
\emph{Required:}
\bibverse{layman_party_2006}
\bibverse{mutz_cross-cutting_2002}

\noindent \emph{Further reading:}
\bibverse{druckman_source_2012}
%\bibverse{stroud_polarization_2010}
\bibverse{conover_political_2011}

\subsection*{Social Media and the Pulse of the Nation}

%\noindent \textbf{OPTIONAL further readings: Social Media and the Pulse of the Nation}
\bibverse[URL TBD. CHAPTER 3.]{herbst_numbered_1995}
\bibverse{kohut_but_2009}
\bibverse{choi_predicting_2012}

\noindent \emph{Further reading:}
\bibverse{oconnor_tweets_2010}
\bibverse{van_dongen_facebook_2012}
\bibverse{chang_how_2010}
\bibverse{messing_election_2012}

\subsection*{The New Activists: Spectacle, Digital Culture, \& Hacktivism}
\subsubsection*{The New Activism}
\emph{Required:}
\bibverse{gladwell_small_2010}
\bibverse{howard_role_2011}

\noindent \emph{Further reading:}
\bibverse{bennett_logic_2012}
\bibverse{bakshy_showing_2013}

\subsubsection*{Digital Protest}
\emph{Required:}
\bibverse{tufekci_social_2012}
\bibverse{esourcevideo_how_2011}

\noindent \emph{Further reading:}
\bibverse{coleman_phreaks_2012}
\bibverse{lim_clicks_2012}

\subsubsection*{Hacktivism}
\emph{Required:}
\bibverse{ludlow_hacktivists_2013}

\noindent \emph{Further reading:}
\bibverse{knappenberger_we_2012}
\bibverse{youmans_social_2012}




%New Media Politics
%In 2008, Barack Obama�s presidential run built on this communicative and organizational momentum by taking advantage of then-emergent social media practices (Harfoush, 2009). 
%
%The Obama campaign, moreover, represented �the fullest realization of trends in the political field toward crafting better means of collecting, storing, analyzing, and acting upon data about citizens, their online behavior, and their social relationships� (Kreiss & Howard, 2010, p. 1033).
%
%Some contend, therefore, that we are living through a �fourth communication revolution� in U.S. history: �an era of information abundance, fracturing the communication monopoly of old-style organizations and allowing many resource-poor new voices to be heard� (Bimber, 2011, p. 7).  
%�decreasing association between the distribution of traditional political resources and the capacity to organize political action� (Bimber, 2011, p. 13).  
%
%Even in the 1990s, well before �Web 2.0� began circulating as a buzzword catchphrase, there was hope that the Internet might level the playing field for long-shot candidates and enrich the depth of our political communication (Klotz, 1997).  
%
%it heralds greater pluralism for marginalized groups and a grassroots renaissance that can undermine elitist monopolies on information (Chadwick, 2006).
%
% how, then, are campaigns being reactive to and proactive about that fragmentation of communicative power in a more interactive era (McNair, 2009, pp. 217, 219)?
%
%�supersede the mediation of television� and are now seeking in the Internet �a new way of detouring� [and] overcoming perceived mass-media obstacles� (Gurevitch, Coleman, & Blumler, 2009, p. 168).  
%
%since most political content still results from �deliberate manipulation by social elites and their �spin industry,�� �skillful communication manipulators can use two-way interactive communication just as effectively as uni-directional communication to steer people� (Louw, 2005, p. 124).  

%Data, Segmentation, and Targeting
%�the art and science of using the available information about the audience, which is to say the work product of all that rating and data-mining, to best advantage� (Manheim, 2011, p. 50).
%
%William McKinley and forerunner to the campaign consultants investigated here, separately targeted �Germans, African Americans, wheelmen, merchants, and even women� for unique speeches and pamphlets (Burton & Shea, 2010, p. 1).  
%
%The strategy of sending different letters to different prospective voters can be traced as far back as Dwight Eisenhower�s election team (Baines, 1999).
%
%Yet it was the 1950s and 1960s when niche-targeting began to take off, with the growth of direct mail advertising representing the strategic precursor (and lingering competitor) to newer online means.  Developing and managing lists of voters became a vital enterprise for aspiring parties eager to exploit this �most precise weapon in a candidate�s appeal� (Johnson, 2001, p. 150); by the 1970s, companies like Claritas were clustering subcultures and lifestyles by zip code in the consumer arena (Burton & Shea, 2010, p. 125).  As cable television fragmented the networks� hold over a nightly mass audience of Americans, the assumption � then as now � was that, �The more specific and tailored the message, the more effective the piece� (Trent et al., 2011, p. 332).
%
%the processing of political information for direct mail, e-mail and Web page customization soon followed (Howard, 2005, p. 8).  
%
%Within a decade, micro-targeting specialists were reaching out to voters based upon statistical probabilities of persuasive effectiveness (Sides, Shaw, Grossmann, & Lipsitz, 2012, p. 127).  
%
%George W. Bush�s 2004 reelection campaign inaugurated many of these tactics to great success, including the synthesis of consumer data with voters records to model and segment more individualized marketing schemes (Issenberg, 2012a); �by examining trends in income, family status, occupation, and other data, the Bush campaign could discover segments of overlooked voters and create a tailored communication strategy to address their needs� (Harfoush, 2009, p. 47).  Offline, this meant, for instance, that volunteers could enter their zip code for a list of party voters in the neighborhood to contact (Graf, 2008, p. 52).
%
%In 2008, Barack Obama�s tech team continue to refine these capacities, including the development of algorithms to predict and address (or avoid) strong positions on hot-button issues (e.g., abortion) during voter outreach (Issenberg, 2012b).  Behavioral targeting began to emerge in this cycle as the Obama campaign exploited online cookies for those who had visited the official Web site and then could receive unique messages about, say, education policy if they clicked onward to a parenting blog (Green, 2008).  
%
%Political parties and specific campaigns now harbor enormous databases of personal information including an individual�s name, address, phone number, voting patterns, political donations, estimated income, race, family members, and even mortgage value and magazine subscriptions (Sides et al., 2012, p. 76).  Aristotle Inc. has been a leading firm in this business of �political data mining,� which also includes the interlocking connections between those individuals featured in its database (Verini, 2007).  Such information has value not only for short-term electoral success, but also long-term strategy (Smith, 2010, p. 140); both parties have contemplated repurposing that data for sale to private vendors or doling it out to favored candidacies in primary contests (Issenberg, 2012c).  With unprecedented access to voters� personal information, strategists now see their work not unlike �the marketing efforts of credit card companies and big-box retailers� and seek to �train voters to go to the polls through subtle cues, rewards and threats� (Duhigg, 2012).  
%
%Consultants/stage managers
%The influence of party bosses declined relative to those �political entrepreneurs� and strategists who might be more loyal to individual candidates (Friedenberg, 1997, p. 23). Various factors prefigured the growth in this multibillion-dollar business, including increasingly necessary skills, the endless duration of campaign cycles, and routinely record-setting infusions of donor cash (p. 199).  At present, an estimated 7,000 professionals now earn all or part of their living on campaigns (Burton & Shea, 2010, p. 9).  

%\vskip.25in
%\noindent\textbf{Important Dates}:
%\begin{center} \begin{minipage}{5in}
%\begin{flushleft}
%Drop Deadline \dotfill Month Day\\
%Add Deadline \dotfill Month Day\\
%First Test \dotfill Month Day\\
%Second Test  \dotfill Month Day\\
%Third Test \dotfill Month Day\\
%Project Deadline \dotfill Month Day\\
%Course Final \dotfill Month Day\\
%\end{flushleft}
%\end{minipage}
%\end{center}


%\begingroup
%\raggedright
%\sloppy
\bibliographystyle{syllabusURLs}
\nobibliography{library}
%\printbibliography
%\endgroup





\end{document}